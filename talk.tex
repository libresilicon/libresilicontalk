\documentclass[9pt]{beamer}
\usetheme{Warsaw}
\usepackage[utf8]{inputenc}
\usepackage[english]{babel}
\usepackage{amsmath}
\usepackage{amsfonts}
\usepackage{amssymb}
\usepackage{graphicx}
\usepackage{tikz}
\usetikzlibrary{
	arrows,
	automata,
        shadings,
        shadows,
        shapes,
}

\input{global_color_scheme.tex}
\input{tikz_process_steps/contsts.tex}

\author{leviathanch | chipforge | foshardware \ (Lanceville Technology)}
\title{Breaking the microchip monopoly}
\setbeamercovered{transparent} 
\setbeamertemplate{navigation symbols}{} 
\logo{lsa.png} 
%\institute{} 
%\date{} 
\subject{A free semiconductor manufacturing standard}
\begin{document}

%\begin{frame}
%\tableofcontents
%\end{frame}

\section[Standard Cells]{}
\begin{frame}{Standard Cells}
\end{frame}

\section[Place'n'Route]{}
\begin{frame}{Open Source Tools}
	\begin{itemize}
        \setlength\itemsep{1em}
		\item graywolf
		\item qrouter
		\item several FPGA routers
	\end{itemize}
\end{frame}

\begin{frame}{graywolf}
	\begin{itemize}
        \setlength\itemsep{1em}
		\item Originates in Academia: TimberWolf
		\item Simulated annealing
	        \begin{itemize}
		    \item Meta heuristic that is useful not only for placement
	        \end{itemize}
		\item Inline syscalls
	        \begin{itemize}
		    \item This is just a bad idea
	        \end{itemize}
	\end{itemize}
\end{frame}

\begin{frame}{qrouter}
	\begin{itemize}
        \setlength\itemsep{1em}
		\item Started in 2011 by Tim Edwards 
		\item Widely used for FPGA
	        \begin{itemize}
		    \item Not ready for silicon
	        \end{itemize}
		\item Sequential routing
	        \begin{itemize}
		    \item Parallelism not in scope
	        \end{itemize}
		\item Difficult to prove formal correctness
	        \begin{itemize}
		    \item Prove that C implementation of Rip-up and Re-route is correct
	        \end{itemize}
	\end{itemize}
\end{frame}

\begin{frame}{Productive Tools}
	\begin{itemize}
        \setlength\itemsep{1em}
		\item Different tool sets like BonnRoute, Cadence, alliance, etc
		\item Similar capabilities with respect to silicon
		\item Just throw man-power at VLSI --- what is automation?
	\end{itemize}
\end{frame}

\begin{frame}{State of the Art}
	\begin{itemize}
        \setlength\itemsep{1em}
		\item Place components for a large chip
		\item Route wires roughly along a chessboard for a large chip
		\item Route detailed tracks and vias for a large chip
		\item Formal correctness: Rip-up and Re-route
		\item Formal style: Sequential/Imperative code
	\end{itemize}
\end{frame}

\begin{frame}{Proposed}
	\begin{itemize}
        \setlength\itemsep{1em}
		\item Decomposition for a large chip
		\item Place components and route for small chips in parallel
		\item Place abstract gates and route recursively
		\item Formal correctness: Reduction from SMT
		\item Formal style: Parallel/Functional code
	\end{itemize}
\end{frame}

\begin{frame}{Divide and Conquer}
	\begin{itemize}
        \setlength\itemsep{2em}
            \item Academia + Industry:
	    \begin{itemize}
            \setlength\itemsep{1em}
		\item Placement and Routing are different problems
		\item All components map to the same problem
	    \end{itemize}
            \item LibreSilicon:
	    \begin{itemize}
            \setlength\itemsep{1em}
		\item Placement and Routing are the same problem
		\item Different components map to different problems
	    \end{itemize}
	\end{itemize}
\end{frame}

\begin{frame}{Parallelism}
	\begin{itemize}
        \setlength\itemsep{1em}
		\item BonnRoute: concurrency + shared memory model
		\item qrouter: none 
		\item lsc: map + reduce
	\end{itemize}
\end{frame}

\begin{frame}{Subcell hierarchies}
	\begin{itemize}
        \setlength\itemsep{1em}
		\item Explicit subcell hierarchies through high modularization
		\item Implicit subcell hierarchies through exlining
		\item Preserve hierarchy in compiler interfaces
	\end{itemize}
\end{frame}

\begin{frame}{High modularization}
       \begin{figure}
        \centering
        \includegraphics[scale=0.38]{SystemBus.png}
       \end{figure}
\end{frame}

\begin{frame}{Exlining}
	\begin{itemize}
        \setlength\itemsep{1em}
		\item Proof of concept: picorv
	\end{itemize}
\end{frame}


\begin{frame}{SMT2}
	\begin{itemize}
        \setlength\itemsep{1em}
		\item Reduction of a *very* common problem and witty problem to SMT
	\end{itemize}
\end{frame}

\begin{frame}{SMT2}
	\begin{itemize}
        \setlength\itemsep{1em}
		\item Show routing related problem in integer programming
	\end{itemize}
\end{frame}

\section[Process]{}

\begin{frame}{Features}
	\begin{itemize}
        \setlength\itemsep{1em}
		\item MOSFETs
		\item LDMOSFETs (High voltage) 
		\item BJTs
		\item Zener polysilicon diodes
		\item SONOS flash cells
		\item Polysilicon resistors
		\item Metal caps
	\end{itemize}
\end{frame}

\begin{frame}{Cross section}
\begin{center}
	\begin{tikzpicture}[node distance = 3cm, auto, thick,scale=0.2, every node/.style={transform shape}]
		\input{tikz_process_steps/glass.a.tex}
		\node at (5,-0.5) {\textbf{\huge{PMOS}}};
		\node at (13,-0.5) {\textbf{\huge{NMOS}}};
		\node at (22,-0.5) {\textbf{\huge{SONOS flash cell (PMOS)}}};
		\node at (30,-0.5) {\textbf{\huge{NPN BJT}}};
		\node at (38,-0.5) {\textbf{\huge{PNP BJT}}};
		\node at (46,-0.5) {\textbf{\huge{Polysilicon diode}}};
		\node at (52,-0.5) {\textbf{\huge{Polyresistor}}};
	\end{tikzpicture}
\end{center}
\end{frame}

\begin{frame}{Photomask}
\begin{center}
\includegraphics[width=0.5\textwidth]{images/20181207_113845_Burst01.jpg}
\end{center}
\end{frame}

\begin{frame}{Photo resist}
\begin{center}
\includegraphics[width=0.4\textwidth]{images/20181128_154907.jpg}
\includegraphics[width=0.4\textwidth]{images/20181128_154911.jpg}
\end{center}
\end{frame}

\begin{frame}{After exposure}
\begin{center}
\includegraphics[width=0.4\textwidth]{images/20181210_125830_Burst01.jpg}
\includegraphics[width=0.4\textwidth]{images/20181210_125845.jpg}
\end{center}
\end{frame}

\begin{frame}{Alignment}
\begin{center}
\includegraphics[width=0.4\textwidth]{images/20181211_125918.jpg}
\includegraphics[width=0.4\textwidth]{images/20181211_161801_Burst01.jpg}
\end{center}
\end{frame}

\end{document}
